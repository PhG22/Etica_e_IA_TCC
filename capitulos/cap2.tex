\chapter{A Concepção do Objeto Técnico}\label{cap:concepcao_obj_tecnico}
 
Este capítulo constrói o alicerce teórico para a análise crítica da concepção de jogos digitais e da Inteligência Artificial, 
conforme proposto na introdução. O objetivo é desmistificar o ato de "conceber um jogo", argumentando que ele transcendeu a criação 
de um mero objeto de entretenimento para se tornar um ato deliberado de engenharia comportamental. A finalidade humana, neste caso, 
a maximização do engajamento e da receita, é inscrita em um objeto técnico que instrumentaliza a psicologia do jogador. A aparente 
"inteligência" dos sistemas de IA que otimizam essa instrumentalização não supera a dialética sujeito-objeto de Álvaro Vieira Pinto \cite{VieiraPinto2005}, 
mas representa sua mais sofisticada forma de ocultamento. Analisaremos como a finalidade corporativa é traduzida em um 
"código comportamental", como esse processo é moldado por relações de poder e, crucialmente, como a narrativa de "jogabilidade 
personalizada" funciona como uma manobra ideológica para obscurecer a intenção manipulativa e o trabalho humano subjacente.

\section{O Trabalho como Fundamento da Técnica e da Hominização}\label{sec:trabalho_como_fundamento}

Para compreender a "concepção de objeto" no contexto tecnológico, é imperativo primeiro estabelecer uma base filosófica que 
transcenda a mera noção de design ou projeto. A concepção de qualquer artefato técnico está fundamentalmente enraizada no conceito 
de trabalho. Na tradição do materialismo histórico, o trabalho não é apenas uma atividade econômica, mas a própria essência da 
atividade humana \cite{Marx2004Manuscritos}, o processo pelo qual o ser humano transforma a natureza para satisfazer suas necessidades e, ao fazê-lo, 
transforma a si mesmo e constrói sua identidade. É através do trabalho que a humanidade se distingue dos outros seres, criando 
cultura, sociedade e história.

Álvaro Vieira Pinto \cite{VieiraPinto2005}, alinhado a essa perspectiva materialista, concebe a técnica não como algo externo ao homem, mas como uma 
extensão dialética de seu próprio ser. A técnica emerge do trabalho como uma mediação entre o homem e a natureza, um prolongamento 
de suas capacidades físicas e intelectuais que permite um domínio crescente sobre o mundo objetivo. Nesta visão, a concepção de um 
objeto técnico não é um ato puramente intelectual ou abstrato, mas uma forma de práxis: a união indissolúvel entre teoria e prática, 
pensamento e ação. Conceber um jogo digital, portanto, não é apenas um exercício de criatividade ou programação,é um ato fundamental 
de trabalho \cite{Marx2004Manuscritos,VieiraPinto2005} que materializa uma intenção humana em um artefato interativo.

\section{A Dialética do Sujeito e do Objeto}\label{sec:dialetica}

Na filosofia da tecnologia de Álvaro Vieira Pinto \cite{VieiraPinto2005}, a dialética entre sujeito e objeto é o ponto de partida para desmistificar a 
tecnologia e situá-la em seu devido lugar, como um produto da existência humana e não como uma força autônoma. O ser humano é o 
único sujeito atuante, um ser histórico dotado de consciência que resolve as contradições de sua existência através do trabalho. 
A máquina, em contrapartida, é, e sempre será, um objeto. Ela é uma mediação, uma extensão da capacidade humana. Vieira Pinto é 
enfático ao afirmar que "a máquina não trabalha". Ela pode executar efeitos dinâmicos, mas o faz como um instrumento passivo, 
seguindo o programa que lhe foi "embutido pelo seu criador, o cérebro humano".

A aplicação direta desta dialética à Inteligência Artificial nos jogos é fundamental. A "inteligência" de um algoritmo de 
\textit{matchmaking} ou de um sistema de precificação dinâmica não é uma propriedade intrínseca do objeto, mas a "exteriorização e a 
multiplicação da racionalidade do sujeito que o concebeu" \cite{VieiraPinto2005}. O discurso corporativo, no entanto, frequentemente inverte essa relação, 
usando frases como "o algoritmo decidiu pareá-lo com este oponente" ou "o sistema personalizou esta oferta para você". Essa inversão 
não é um erro linguístico inocente, é o primeiro passo de uma sofisticada mistificação ideológica. Ao antropomorfizar o objeto, 
esvazia-se a responsabilidade do sujeito humano, o engenheiro, o designer de monetização, a corporação. Se o algoritmo é o sujeito 
que "decide", então o criador humano é absolvido da responsabilidade pelas consequências dessa decisão, seja uma experiência de jogo 
frustrante projetada para induzir gastos ou uma oferta predatória direcionada a um jogador vulnerável. A própria linguagem usada para 
descrever a IA nos jogos torna-se um campo de batalha ideológico, preparando o terreno para o ocultamento do poder e da intenção por 
trás do sistema.

\section{A Natureza do Objeto Técnico}\label{sec:natureza_obj_tec}

A visão comum da tecnologia tende a reduzi-la a um conjunto de ferramentas neutras. A Teoria Crítica da Tecnologia de Andrew Feenberg \cite{Feenberg1999}
desafia essa concepção, argumentando que a tecnologia nunca é neutra, ela é um palco de valores sociais e políticos. Todo artefato 
técnico possui um "código técnico", um conjunto de regras e pressupostos embutidos em seu design que refletem e reforçam uma 
determinada visão de mundo e específicas relações de poder \cite{Feenberg2002Transforming}. A escolha de design entre múltiplas possibilidades tecnicamente viáveis é 
sempre política.

No contexto dos jogos digitais potencializados por IA, esse conceito pode ser aprofundado. Conforme argumentam Jong e Prey \cite{deJongPrey2022}, o código 
técnico de muitos sistemas algorítmicos contemporâneos é, fundamentalmente, o Behaviorismo. Esta é uma visão de mundo que reduz o ser 
humano a um conjunto de comportamentos observáveis, previsíveis e controláveis, ignorando a complexidade da subjetividade. Este 
trabalho postula que o código técnico dos sistemas de IA em jogos digitais é, precisamente, um "código comportamental". Os algoritmos 
de monetização e engajamento não são neutros, eles codificam uma visão empobrecida e instrumental do ser humano, tratando o jogador 
não como um sujeito a ser entretido, mas como um sistema de respostas a ser otimizado para a extração de valor. A engenharia de 
software, neste contexto, transcende a construção de funcionalidades para se tornar a codificação de uma epistemologia behaviorista 
que serve a fins comerciais.

\section{A Inscrição da Finalidade Comercial no Código Comportamental}\label{sec:inscricao_finalidade}

Em sistemas de IA, a inscrição da finalidade assume uma forma diferente da programação tradicional. Ela se materializa como a 
arquitetura de um vasto espaço de possibilidades, moldado pelos dados de treinamento e pelos objetivos de otimização. Nos jogos 
digitais, o ato de "conceber" um sistema de monetização com IA e de lhe atribuir a finalidade de "maximizar a receita" torna-se o ato 
de projetar um sistema que aprende e se adapta para explorar a psicologia do jogador.

A finalidade comercial é inscrita através de:

\begin{itemize}
    \item \textbf{Testes A/B em Larga Escala:} Sistemas de IA podem executar milhares de testes simultaneamente, variando preços, ofertas e a 
    apresentação visual de itens para diferentes segmentos de jogadores \cite{Lupianez-Villanueva2022}, identificando e aplicando automaticamente as táticas que 
    geram maior conversão. 
    
    \item \textbf{Precificação Dinâmica:} Algoritmos analisam o histórico de compras, o tempo de jogo e outros dados comportamentais de um 
    jogador para calcular o preço máximo que ele estaria disposto a pagar por um item e apresentar uma oferta "personalizada" nesse 
    exato valor.
    
    \item \textbf{Geração Procedural de Conteúdo para Retenção:} A IA pode gerar desafios, missões ou mapas adaptados ao nível de habilidade 
    do jogador, não para maximizar a diversão, mas para mantê-lo em um estado de engajamento ótimo que o torne mais propenso a gastar.
\end{itemize}

A natureza estatística e opaca desses sistemas, o chamado "problema da caixa-preta", é uma poderosa ferramenta ideológica. Ela 
permite que decisões de design manipulativas sejam apresentadas como resultados "otimizados" e "emergentes" do processo de 
aprendizado da máquina \cite{Bender2021}, em vez de escolhas deliberadas de seus criadores. A ideologia da extração de valor é, assim, "lavada" através 
da estatística, transformando decisões de negócio em resultados aparentemente neutros e técnicos.

\section{A Concepção como Arena Sociopolítica}\label{sec:concepcao_arena_sociopolitica}

A concepção de um jogo não ocorre em um vácuo. Ela está imersa em contextos sociais e políticos que moldam sua forma e função. A 
teoria da Construção Social da Tecnologia, \textit{SCOT} \cite{BijkerPinch1989} argumenta que o design é negociado por diferentes "grupos sociais relevantes". 
Na indústria de jogos, esses grupos incluem não apenas desenvolvedores e jogadores, mas também executivos, investidores, equipes de 
marketing e, cada vez mais, especialistas como \textit{game economists} e designers de monetização.

A finalidade inscrita nos jogos reflete os interesses do grupo dominante, o capital. A pressão corporativa por metas de receita 
trimestrais impulsiona a adoção de mecânicas de monetização agressivas e \textit{dark patterns}, mesmo que prejudiquem a experiência 
do jogador a longo prazo \cite{PetrovskayaZendle2022}. O designer ou engenheiro, neste contexto, muitas vezes atua não como um criador autônomo, mas como um 
trabalhador cuja criatividade é direcionada para resolver o problema de "como extrair mais valor do jogador". Este processo depende 
de uma cadeia global de trabalho, desde os desenvolvedores em regime de \textit{crunch} até os testadores precarizados, cujo trabalho 
coletivo para criar o jogo é sistematicamente ocultado por trás da marca do estúdio.

\section{A Mistificação do Objeto}\label{sec:mistificacao_obj}

A crítica de Álvaro Vieira Pinto ao conceito de "tecno-estrutura" de Galbraith \cite{VieiraPinto2005,Galbraith1967} oferece um análogo preciso para compreender a 
mistificação contemporânea da IA nos jogos. Galbraith argumentava que o poder havia se deslocado do capital para os "técnicos". 
Vieira Pinto desmonta essa noção, mostrando que o conhecimento técnico é apenas uma mercadoria comprada pelo capital, e a 
"tecno-estrutura" é uma fachada ideológica para mascarar a dominação inalterada dos proprietários.

A narrativa do jogo "inteligente" ou "adaptativo" desempenha a mesma função. Ao atribuir agência ao sistema de IA, que "aprende" o 
comportamento do jogador e "personaliza" a experiência, desvia-se a atenção dos verdadeiros agentes, as corporações que detêm os 
modelos e os dados, e que os controlam para fins de lucro. A IA torna-se a nova tecno-estrutura, uma fachada de objetividade técnica 
que oculta as relações de produção e a finalidade de manipulação. Essa mistificação torna invisível tanto o trabalho precarizado que 
constrói o jogo \cite{GraySuri2019} quanto a intenção deliberada de engenharia comportamental que o anima, transformando uma relação de exploração 
econômica em uma "funcionalidade" de software.

\section{A Reafirmação da Agência Humana na Lei e no Trabalho}\label{sec:reafirmacao_agencia_humana}

A crescente sofisticação dos sistemas de IA em jogos, alimentada pelo discurso de sua autonomia, provocou uma reação social e 
institucional que busca reafirmar a centralidade e a responsabilidade humanas. Este movimento contesta na prática a mistificação do 
objeto.

Na esfera regulatória, o \textit{Al Act} da União Europeia \cite{EU_AI_Act2024}, que proíbe o uso de "técnicas subliminares que ultrapassem a consciência 
de uma pessoa" para manipular o comportamento , e decisões como a da Autoridade Holandesa para Consumidores e Mercados, a \textit{ACM} 
contra a \textit{Epic Games} \cite{ACM2024EpicGames}, são exemplos claros. A multa à \textit{Epic} por usar \textit{dark patterns} como temporizadores de 
contagem regressiva falsos em \textit{Fortnite} não penalizou um algoritmo autônomo, mas responsabilizou a empresa humana por suas 
escolhas de design predatórias, especialmente contra crianças. Essas ações forçam legalmente o restabelecimento do ser humano como o 
sujeito soberano e a IA como o objeto que serve aos seus propósitos \cite{EU_AI_Act2024}.

No campo do trabalho, a greve da \textit{Writers Guild of America - WGA} em 2023, que garantiu que a IA não pudesse ser creditada 
como "escritora" \cite{WGA_MBA2023} , ecoa as preocupações da indústria de jogos. A luta é para que a concepção da obra, seja um roteiro ou um jogo, 
permaneça uma prerrogativa humana, relegando a IA à sua posição correta de ferramenta. Essas lutas regulatórias e trabalhistas são a 
prova de que a dialética sujeito-objeto é o cerne da política tecnológica contemporânea, forçando a reafirmação da agência e da 
responsabilidade humana contra a narrativa de autonomia da máquina.

\section{Consciência Crítica na Concepção}\label{sec:consciencia_critica_concepcao}

A análise desenvolvida neste capítulo demonstra que a concepção de um jogo digital potencializado por IA, longe de ser um processo 
técnico neutro, é um ato profundamente humano, social e político. A partir do referencial de Álvaro Vieira Pinto \cite{VieiraPinto2005}, foi estabelecido 
que a relação fundamental é a de um sujeito que cria um objeto de mediação para satisfazer uma finalidade. Nos jogos modernos, essa 
finalidade é crescentemente a extração de valor, inscrita em um "código comportamental" \cite{deJongPrey2022} que visa modular a psicologia do jogador.

A narrativa de autonomia da IA foi desmistificada como uma fachada ideológica que oculta o poder corporativo e o trabalho humano \cite{VieiraPinto2005}. As 
implicações para a engenharia são profundas. Uma prática de engenharia consciente não pode se esconder atrás da suposta objetividade 
do artefato. Ela deve reconhecer sua posição como agente no processo de concepção, questionando as finalidades impostas e assumindo 
a responsabilidade pela mediação que seus objetos criam no mundo. A verdadeira "inteligência" não reside no artefato, mas na 
consciência crítica do sujeito que o concebe e na coletividade de sujeitos cujo trabalho o torna possível. Alinhar-se a uma 
engenharia centrada no humano significa assumir o papel de sujeito transformador, e não o de mero executor em uma linha de montagem 
industrial digital.

