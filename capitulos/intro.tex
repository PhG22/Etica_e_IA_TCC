\chapter{Introdução}\label{cap:introducao}

Neste capítulo pretende-se introduzir brevemente o leitor aos temas pertinentes ao trabalho.
Serão apresentados o tema, seu domínio, objeto e pergunta de pesquisa e 
implicações na computação.
\section{Considerações Iniciais}\label{sec:consideracoes_iniciais}

A indústria de desenvolvimento de software, frequentemente celebrada como um bastião da inovação, criatividade e flexibilidade, opera sobre um paradoxo fundamental. 
Por um lado, 
a concepção e a construção de constructos computacionais são atividades eminentemente intelectuais, que demandam engenhosidade, resolução de problemas complexos e um grau de  
artesanato digital. 
Movimentos como o "\textit{Software Craftsmanship}", Artesanato de Software, surgiram como uma resposta à industrialização do desenvolvimento, enfatizando a 
importância da qualidade, do profissionalismo e do orgulho no trabalho bem-feito, onde o desenvolvedor é visto como um artesão que aprimora suas habilidades e cria produtos de 
alta qualidade.

Por outro lado, esta atividade criativa é, em sua vasta maioria, exercida no interior de estruturas corporativas que aplicam lógicas de produção industrial, buscando 
previsibilidade, padronização e maximização da eficiência. 
Essa abordagem, herdeira da gestão científica de Frederick Taylor, o Taylorismo \cite{Taylor1911}, visa decompor tarefas complexas em 
partes menores e gerenciáveis, otimizar fluxos de trabalho e, idealmente, tornar o trabalhador individual uma peça substituível em uma linha de montagem. 
No desenvolvimento de 
software, isso se manifesta na tentativa de transformar a programação em um processo previsível e mensurável, muitas vezes em detrimento da autonomia e da criatividade do 
engenheiro.

Este trabalho de conclusão de curso se propõe a investigar a tensão inerente a este paradoxo. 
A aplicação de modelos industriais a um trabalho de natureza criativa gera uma 
forma de alienação, conceito analisado por Karl Marx \cite{Marx1844}. 
O trabalhador de software, muitas vezes, encontra-se alienado do produto final de seu trabalho, executando tarefas 
fragmentadas sem uma visão do todo, 
do próprio ato de produção, seguindo processos sobre os quais tem pouco controle, de seus colegas, devido à especialização excessiva, 
e de 
sua própria essência criativa, ao ser reduzido a um executor de tarefas pré-definidas.

Nenhum campo exemplifica melhor essa tensão e suas consequências do que a Inteligência Artificial. 
A IA é o auge da narrativa industrial, a promessa de uma automação completa, 
onde a "inteligência" reside na própria máquina. 
Contudo, essa é uma fachada que oculta uma vasta e diversificada gama de trabalho humano. 
Por trás da aparente autonomia dos 
algoritmos, existe um exército global de "trabalhadores fantasma" (\textit{ghost workers}) \cite{GraySuri2019}, que realizam microtarefas de rotulagem de dados, moderação de conteúdo e correção de erros dos 
sistemas, muitas vezes em condições precárias e por remuneração mínima. 
Esse processo representa a taylorização levada ao extremo, o trabalho cognitivo humano é fragmentado em 
suas unidades mais básicas para "inteligenciar" a máquina, tornando o trabalhador invisível. 

Essa mistificação da IA cumpre uma função ideológica análoga àquela criticada por Álvaro Vieira Pinto \cite{VieiraPinto2005} em sua análise do conceito de "tecno-estrutura" de John K. Galbraith \cite{Galbraith1967}. 
Assim como a ideia de que 
o poder se deslocou do capital para os "técnicos" servia para ocultar a dominação inalterada dos proprietários, a narrativa da "IA inteligente" mascara o poder das corporações 
que detêm os modelos e os dados, e apaga a centralidade do trabalho humano que a constitui. 
A máquina, que na filosofia de Vieira Pinto é um objeto de mediação da ação humana, 
é apresentada como um sujeito autônomo, invertendo a relação fundamental entre criador e criação. 

\section{Domínio}\label{sec:dominio}

Esta pesquisa se situa na interseção entre a Engenharia de Software, a Sociologia do Trabalho e a Teoria Crítica da Tecnologia. 
O domínio abrange a análise dos processos de 
desenvolvimento de software sob a ótica das teorias de organização industrial, como o Taylorismo, e suas críticas, com um foco particular na aplicação desses modelos à produção 
de sistemas de Inteligência Artificial. 
Utiliza-se como referencial teórico a filosofia da tecnologia de Álvaro Vieira Pinto para desmistificar a autonomia da técnica e 
reafirmar a centralidade do sujeito humano, bem como estudos sobre o trabalho digital e o "trabalho fantasma" na economia de plataforma. 

\section{Objeto e Pergunta de Pesquisa}\label{sec:obj_pergunta_pesquisa}

O objeto desta pesquisa é o processo de concepção e produção de constructos computacionais, com ênfase nos sistemas de Inteligência Artificial. 
A investigação se concentra na 
análise do trabalho humano — tanto o trabalho intelectual e criativo dos engenheiros de software quanto o trabalho fragmentado e muitas vezes invisibilizado dos trabalhadores 
de dados, que é sistematicamente ocultado pelos modelos de produção industrial e pelas narrativas de automação que dominam a indústria de tecnologia. 

A partir da tensão exposta, a pergunta que guia esta monografia é: De que maneira os modelos de produção industrial aplicados ao desenvolvimento de software, especialmente na 
área de Inteligência Artificial, geram uma tensão com a natureza criativa do trabalho e resultam no ocultamento do trabalho humano essencial para a sua realização? 

\section{Implicações na Computação}\label{sec:implicacoes_computacao}

As implicações desta pesquisa para a área de Engenharia de Computação são, primordialmente, de ordem crítica e ética. 
Ao desvelar o paradoxo no cerne da produção de software, 
este trabalho desafia a visão puramente técnica e instrumental da engenharia. 
Ele convida os futuros engenheiros a refletirem sobre as seguintes questões: 

\begin{itemize}
    \item \textbf{A Natureza do Trabalho de Engenharia:} Reconhecer a engenharia de software não apenas como uma disciplina técnica, mas como uma prática criativa e intelectual que é impactada e, 
por vezes, limitada por modelos de gestão.
\item \textbf{A Responsabilidade Ética e Social:} Compreender que as escolhas de arquitetura e processo não são neutras. 
Elas estão inseridas em um sistema de produção que tem consequências 
diretas sobre as condições de trabalho de uma vasta cadeia de pessoas, desde os engenheiros na empresa até os trabalhadores de dados em plataformas globais. 
\item \textbf{A Necessidade de uma Prática Crítica:} Incentivar uma postura que questione a finalidade dos sistemas que são construídos e os interesses que eles servem, alinhando-se a uma 
engenharia que, conforme a perspectiva de Vieira Pinto, deve ser centrada no humano e consciente de seu papel como sujeito transformador, e não como mero executor em uma linha 
de montagem industrial. 
\end{itemize}