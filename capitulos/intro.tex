\chapter{Introdução}\label{cap:introducao}

Neste capítulo pretende-se introduzir brevemente o leitor aos temas pertinentes ao trabalho.
Serão apresentados o tema, seu domínio, objeto e pergunta de pesquisa e 
implicações na computação.

\section{Considerações Iniciais}\label{sec:consideracoes_iniciais}

A indústria de desenvolvimento de software, frequentemente celebrada como um bastião da inovação, criatividade e flexibilidade, 
opera sobre um paradoxo fundamental. Por um lado, a concepção e a construção de constructos computacionais são atividades 
eminentemente intelectuais, que demandam engenhosidade, resolução de problemas complexos e um grau de artesanato digital. 
Movimentos como o \textit{"Software Craftsmanship"} surgiram como uma resposta à industrialização do desenvolvimento, enfatizando a 
importância da qualidade, do profissionalismo e do orgulho no trabalho bem-feito, onde o desenvolvedor é visto como um artesão que 
aprimora suas habilidades e cria produtos de alta qualidade. Por outro lado, esta atividade criativa é, em sua vasta maioria, 
exercida no interior de estruturas corporativas que aplicam lógicas de produção industrial, buscando previsibilidade, padronização e 
maximização da eficiência. Essa abordagem, herdeira da gestão científica de Frederick Taylor, o Taylorismo \cite{Taylor1911}, visa decompor tarefas 
complexas em partes menores e gerenciáveis, otimizar fluxos de trabalho e, idealmente, tornar o trabalhador individual uma peça 
substituível em uma linha de montagem.

Um setor em que essa tensão se manifesta de forma aguda e com vastas consequências é a indústria de jogos digitais. Com uma 
receita global que supera as indústrias de cinema e música combinadas, os jogos se tornaram um campo central da produção cultural e 
tecnológica contemporânea. No entanto, por trás da fachada de entretenimento e criatividade, as condições de trabalho dos 
desenvolvedores frequentemente refletem a face mais severa da industrialização. Práticas como o \textit{crunch}, períodos de trabalho 
excessivo e não remunerado antes do lançamento de um produto, e a crescente precarização, especialmente para testadores de 
qualidade, os \textit{QA testers}, que são frequentemente contratados e demitidos em ciclos curtos, demonstram que a lógica taylorista 
de extração máxima de produtividade persiste e se intensifica. 

Contudo, a contribuição mais significativa e perturbadora da indústria de jogos para este paradoxo é a sua expansão para além do 
chão de fábrica digital. A mesma lógica de fragmentação e otimização, antes aplicada ao trabalhador, é agora metodicamente voltada 
contra o próprio consumidor. A experiência do jogador é decomposta em "loops de engajamento" e micro-objetivos, cuidadosamente 
projetados não para maximizar a diversão de forma intrínseca, mas para otimizar métricas de retenção e monetização. Este processo 
representa a aplicação sistemática dos princípios do Behaviorismo Radical de B. F. Skinner \cite{DelpratoMidgley1992}, onde o jogador é condicionado através de 
esquemas de reforço intermitente, recompensas variáveis e imprevisíveis, para manter o engajamento e incentivar gastos recorrentes.

Essa dinâmica expande o conceito de alienação, analisado por Karl Marx \cite{Marx1844}, para além do trabalhador e o estende ao jogador. O engenheiro 
de software encontra-se alienado do produto final, do processo criativo e de sua própria e  ssência ao ser reduzido a um executor de 
tarefas pré-definidas. De forma análoga, o jogador é sutilmente alienado de sua própria agência lúdica. Suas decisões, que ele 
percebe como livres e autônomas, são na verdade guiadas e moduladas por sistemas opacos. Algoritmos de \textit{matchmaking}, por exemplo, 
podem ser projetados não para criar partidas justas, mas para otimizar o engajamento, potencialmente manipulando a experiência para 
incentivar o jogador a continuar jogando ou a realizar uma compra. Sistemas de precificação dinâmica, por sua vez, podem ajustar o 
custo de itens virtuais em tempo real com base no perfil de comportamento do jogador, extraindo o máximo valor possível. O jogador, 
imerso no ato de jogar, raramente percebe que sua experiência é um campo de experimentação e extração econômica.

A Inteligência Artificial emerge como a tecnologia catalisadora que permite a aplicação dessas lógicas em uma escala massiva, 
personalizada e em tempo real. A IA não é apenas o produto final que, como se argumentará, oculta o trabalho humano em sua concepção; 
ela é, crucialmente, a ferramenta que implementa e otimiza os mecanismos de manipulação comportamental. Sistemas de IA são usados 
para potencializar a eficácia de \textit{dark patterns}, padrões de design enganosos que levam o usuário a tomar decisões contra seus 
próprios interesses \cite{Brignull2023Deceptive}, e para refinar estratégias de monetização predatória, que exploram vieses cognitivos para induzir gastos.

Assim, a crítica ao ocultamento do trabalho na era da IA deve ser aprofundada. O problema não se resume à invisibilidade do 
"trabalho fantasma", o conhecido \textit{ghost work} \cite{GraySuri2019},a rotulagem de dados e a moderação de conteúdo que treinam os modelos de IA. 
Nos jogos digitais, testemunha-se uma dupla ocultação. A primeira camada é a do trabalho de produção, o trabalho precarizado dos 
desenvolvedores e dos trabalhadores de dados que constituem a base material do sistema. A segunda camada, igualmente importante, é a 
do trabalho de design manipulativo. A intenção humana de modular o comportamento do jogador, codificada nos algoritmos, é 
sistematicamente ocultada sob a  fachada de uma jogabilidade "personalizada", "dinâmica" ou "imersiva". A IA, na sua mais sofisticada 
mistificação, oculta tanto sua base material, o trabalho, quanto sua finalidade ideológica, a manipulação, representando o ápice da crítica 
de Álvaro Vieira Pinto à tecnologia como um objeto que mascara as relações de poder \cite{VieiraPinto2005}.

\section{Domínio}\label{sec:dominio}

Esta pesquisa se situa na interseção entre a Engenharia de Software, a Sociologia do Trabalho e a Teoria Crítica da Tecnologia. No 
entanto, a análise aprofundada do objeto "IA em jogos digitais" exige a expansão deste domínio para incorporar campos de conhecimento 
que são essenciais para decodificar a complexidade do fenômeno. uma melhor compreensão requer a integração de áreas como:

\begin{itemize}
    \item \textbf{Estudos Críticos de Jogos (\textit{Critical Game Studies}):} Este campo permite analisar o jogo não como um 
    software neutro ou um mero produto de entretenimento, mas como um artefato cultural e político. Os jogos refletem, reproduzem e, 
    por vezes, contestam as ideologias e relações de poder da sociedade em que são criados. Uma perspectiva crítica permite 
    investigar como os jogos podem normalizar lógicas de trabalho precarizado, consumo excessivo e vigilância de dados sob o véu do 
    lúdico. 

    \item \textbf{Economia Comportamental e Psicologia Behaviorista:} A compreensão dos mecanismos de monetização e engajamento nos 
    jogos modernos é impossível sem recorrer a estas disciplinas. O design de muitos sistemas, como as \textit{loot boxes}, 
    caixas de recompensa aleatórias, e as microtransações \cite{GordonPetrovskaya2023}, é fundamentado em princípios de psicologia comportamental para explorar 
    vieses cognitivos humanos, como a aversão à perda, o viés de custo afundado e a sensibilidade a esquemas de recompensa variável. 
    A monetização predatória \cite{PetrovskayaZendle2022}, que visa extrair o máximo de receita de uma pequena fração de jogadores, é a aplicação direta desses 
    princípios. A análise destes sistemas revela uma engenharia deliberada da psicologia do jogador.

    \item \textbf{Ética do Design:} Este campo fornece o arcabouço para avaliar a responsabilidade moral dos engenheiros, designers 
    e corporações. A implementação de \textit{dark patterns} \cite{Lupianez-Villanueva2022} e mecânicas viciantes não é um acidente técnico, mas o resultado de 
    decisões de design tomadas sob intensa pressão corporativa por resultados financeiros. A ética do design questiona os limites 
    entre persuasão e manipulação, e investiga como criar experiências digitais que respeitem a autonomia e o bem-estar do usuário, 
    em vez de explorá-lo. 
\end{itemize}

A articulação desses domínios permite uma análise mais precisa da natureza da tecnologia em questão.

\section{Objeto e Pergunta de Pesquisa}\label{sec:obj_pergunta_pesquisa}

A partir da expansão do domínio e do aprofundamento da crítica, torna-se necessário refinar o escopo da investigação para garantir 
precisão e relevância. A generalidade de "constructos computacionais" é substituída por um foco nos mecanismos específicos que 
exemplificam a tese central.

O objeto deste trabalho é o processo de concepção e implementação de sistemas de IA em jogos digitais, com foco em mecanismos de 
monetização predatória, como precificação dinâmica, loot boxes, entre outros \cite{PetrovskayaZendle2022}, modulação de comportamento, como os\textit{dark patterns} \cite{Brignull2023Deceptive}
e otimização de engajamento, como o \textit{Engagement-Optimized Matchmaking}, que são sustentados por trabalho humano oculto e visam 
a instrumentalização do jogador para extração de valor.

A partir da tensão exposta, a pergunta que guia esta monografia é: De que maneira a aplicação de lógicas de produção industrial e de 
economia comportamental, potencializada por sistemas de Inteligência Artificial, transforma a concepção de jogos digitais, resultando 
tanto no ocultamento do trabalho humano que os sustenta quanto na instrumentalização da subjetividade do jogador para fins de 
extração de valor?

\section{Implicações na Computação}\label{sec:implicacoes_computacao}

As implicações desta pesquisa para a área de Engenharia de Computação são, primordialmente, de ordem crítica e ética, movendo a 
discussão de um plano abstrato para um plano concreto e urgente. O engenheiro de computação que projeta um algoritmo de precificação 
dinâmica, um sistema de \textit{loot boxes} ou um algoritmo de \textit{matchmaking} não está apenas otimizando um sistema técnico, 
ele está ativamente arquitetando um ambiente projetado para influenciar o comportamento humano. Suas escolhas de design, muitas vezes 
guiadas por métricas de negócio como retenção, conversão e receita média por usuário pagante, têm consequências éticas diretas. Ele 
assume, conscientemente ou não, o papel de um \textit{game economist} ou de um designer de monetização, cujas decisões podem promover 
bem-estar ou exploração. 

O caso da multa de 1.125.000 euros aplicada pela Autoridade Holandesa para Consumidores e Mercados, a \textit{ACM} à 
\textit{Epic Games}, desenvolvedora de \textit{Fortnite} \cite{ACM2024EpicGames}, serve como um exemplo paradigmático. A empresa foi penalizada por práticas 
comerciais desleais direcionadas a crianças, incluindo o uso de frases como "Compre agora" e, crucialmente, temporizadores de contagem 
regressiva enganosos para criar uma falsa sensação de escassez, um tipo clássico de \textit{dark pattern}. Este caso demonstra que as 
implicações da engenharia comportamental não são meramente teóricas, elas são legais, financeiras e têm impacto direto sobre 
populações vulneráveis. A decisão da \textit{ACM} não aponta para um "erro" de software, mas para uma escolha deliberada de design, 
o resultado lógico de um processo de concepção que priorizou a maximização da receita em detrimento da ética e do bem-estar do jogador. 

Este trabalho desafia a visão puramente técnica e instrumental da engenharia e convida a refletir sobre a necessidade de desenvolver uma 
"consciência lúdico-crítica". Esta postura vai além da competência técnica e envolve:

\begin{itemize}
    \item \textbf{Alfabetização em Design Ético:} A capacidade de reconhecer, nomear e analisar criticamente \textit{dark patterns}, \cite{LeiserSantos2024}
    mecânicas de monetização predatória e outras formas de design manipulativo, compreendendo os princípios psicológicos que elas 
    exploram.
    
    \item \textbf{Questionamento da Métrica:} Desafiar a otimização de métricas de negócio como o único ou principal objetivo do design. 
    Isso implica em perguntar não apenas "este sistema aumenta a conversão?", mas também "este sistema respeita a autonomia do jogador?" e 
    "quais são seus potenciais danos?".
    
    \item \textbf{Visibilidade do Trabalho:} Assumir a responsabilidade ética de dar crédito e visibilidade a todas as formas de 
    trabalho que contribuem para o sistema, desde os artistas, programadores e testadores, até os trabalhadores de dados \cite{GraySuri2019}, muitas 
    vezes em condições precárias, que treinam os modelos de IA subjacentes. 
\end{itemize}

Em última análise, este trabalho se alinha à perspectiva de Álvaro Vieira Pinto \cite{VieiraPinto2005}, que define a tecnologia como uma mediação da ação 
humana no mundo. Nos jogos digitais potencializados por IA, essa mediação foi transformada em uma ferramenta de engenharia 
comportamental em massa, operando em uma escala sem precedentes. O desafio para a Engenharia de Computação é decidir se continuará a 
ser uma executora acrítica dessa lógica, escondendo-se atrás da suposta neutralidade da técnica, ou se assumirá seu papel como 
sujeito transformador, consciente de seu poder e responsabilidade para projetar sistemas que respeitem a agência e a dignidade tanto 
daqueles que os criam quanto daqueles que os jogam. 

